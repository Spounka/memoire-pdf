\section{Introduction général}

Le monde connaît un développement très rapide au niveau de technologie informatique. La gestion informatisée de la plupart des systèmes qui simplifie la majorité de nos taches quotidienne, mais malheureusement y a certain certains domaines non touchés\\
``Connais-tu un bon plombier?''
\hfill
ou
\hfill
``Si quelqu’un cherche un maçon, contacte-moi''.

\noindent Ce sont des phrases trop communes dans notre société, un artisan qui cherche un client pour gagner un peux d'argent ou une personne ayant un problème de soudure ou problème de plomberie et elle cherche un artisan qualifié pour le résoudre.\\

Durant notre vie quotidienne, on a souvent besoins de certains services de bricolage (remplacer une prise ou placer un chauffage central) mais malheureusement le temps passé à la recherche d’un bon artisans n’est pas court et parfois en se fait trompé par un artisan qui fais seulement un demis travail ou bien aucun travail.
\subsection*{Solution}
Une plateforme dans laquelle un utilisateur (client) peut trouver et contacter un artisan spécialisé dans le domaine requis sans énormes efforts et recherches qui durent longtemps.
Dans cette plateforme, un visiteur sera présenté par une liste des artisans disponible avec touts détailles nécessaires comprenant le prix, l’adresse, le numéro de téléphone, le nom et le prénom, ensuite le visiteur peut choisir de contacter un artisans- directement en utilisant le numéro de téléphone ou bien s’inscrire pour lui contacter dans la plateforme et discuter les détailles du projet.

Ce travail sera divisé en trois chapitres:
\begin{enumerate}
	\item Le premier chapitre sera dédié pour l’analyse des besoins, le cahier des charges, l'identification des acteurs et enfin la liste des besoins fonctionnels et non-fonctionnels.
	\item Le deuxième chapitre contient la conception du projet qui consiste des diagrammes de séquences, diagrammes d’activés, diagramme de classe final et en fin le schéma de la base de données.
	\item Le troisième chapitre contient des détailles sur l’implémentation et la réalisation de l’application ainsi qu’une petite discussion sur les technologies utiliséesLe premier chapitre sera dédié pour l’analyse des besoins, le cahier des charges, l’identification des acteurs et enfin la liste des besoins fonctionnels et non-fonctionnels
\end{enumerate}
