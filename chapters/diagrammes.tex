\pagebreak
\subsection{Cas d'utilisation Crée compte}
\renewcommand{\arraystretch}{1.3}
\begin{center}
	\begin{table}[H]
			\centering
			\tiny\begin{tabular}{ | l | m{0.51\textheight} |}
			\hline
			\rowcolor[HTML]{06a8ed}
			\textbf{Nom du cas} & Crée un compte \\
			\hline\hline
			\cellcolor[HTML]{99ccff} \textbf{Type} & Principal\\
			\hline
			\cellcolor[HTML]{99ccff} \textbf {Acteur Principal} & Visiteur\\
			\hline
			\cellcolor[HTML]{99ccff} \textbf Objective & Permet à tout visiteur du site de	crée un compte et devenir client\\
			\hline
			\cellcolor[HTML]{99ccff}\textbf {Pré-condition} & Consulter Site\\
			\hline
			\cellcolor[HTML]{99ccff} \textbf {Scénario Nominal} & \parbox{0.43\textheight}{
					\begin{enumerate}
						\vspace{0.01\textheight}
						\item Le visiteur clique << Crée Compte >>.
						\item Le site affiche le formulaire du création des comptes.
						\item Le visiteur remplis les champs affiché.
						\item Le visiteur confirme les informations.
						\item Le système vérifie informations.
						\item Le système crée un compte pour l'utilisateur et le redirige vers son profil.
						\vspace{0.01\textheight}
					\end{enumerate}}\\
			\hline
			\cellcolor[HTML]{99ccff} \textbf {Scénario Alternatif} & \parbox{0.43\textheight}{
				\begin{enumerate}
					\item \textbf{A1: Les informations du visiteur ne sont pas valides:}
						\subitem 5. Le site affiche message d'erreur.\newline						
					La séquence résume du point 2.
					
					\item \textbf{A2: Le compte existe déjà:}
						\subitem 1. Le système demande à l'utilisateur de modifier l'identifiant ou de se connecter\newline
					Le séquence reprend au point 2
				\end{enumerate}}\\
			\hline
			\cellcolor[HTML]{99ccff} \textbf{Scénario d'exception} & --- \\
			\hline
			\cellcolor[HTML]{99ccff} \textbf{Post-condition} & Un nouveau compte client sera crée\\
			\hline
		\end{tabular}
		\caption{Description Crée compte}
		\label{table:create account}
	\end{table}
	\begin{figure}[H]
		\centering
		\includegraphics[scale=0.4]{CreateAccount.jpg}
		\caption{Diagramme séquence Crée un compte}
		\label{fig:seq create account}
	\end{figure}
\end{center}


\subsection{Cas d'utilisation Gérer profil (modifier)}
\renewcommand{\arraystretch}{2}
\begin{center}
	\begin{table}[H]
		\centering
		\tiny{\begin{tabular}{ | l | m{0.51\textheight} |}
			\hline
			\rowcolor[HTML]{06a8ed}
									 \textbf{Nom du cas} & Gérer profil (modifier)\\ 
			\hline \hline
			\cellcolor[HTML]{99ccff} \textbf{Type} & Principal\\
			\hline
			\cellcolor[HTML]{99ccff} \textbf{Acteur Principal} & Client, Artisans\\
			\hline
			\cellcolor[HTML]{99ccff} \textbf{Objective} & Permet à l'acteur de modifier les informations de son compte\\
			\hline
			\cellcolor[HTML]{99ccff} \textbf{Pré-condition} & S'authentifier\\
			\hline
			\cellcolor[HTML]{99ccff} \textbf{Scénario Nominal} & \parbox{0.43\textheight}{
					\begin{enumerate}
						\vspace{0.01\textheight}
						\item L'acteur clique << Gérer Profile >>.
						\item Le site affiche toutes informations cecernant l'acteur.
						\item L'acteur peut modifier les informations affichées.
						\item L'acteur confirme les modifications.
						\item Le système vérifie informations.
						\item Le système enregistre les informations.
						\item Le système affiche message de succèss.
						\vspace{0.01\textheight}
					\end{enumerate}}\\
			\hline
			\cellcolor[HTML]{99ccff} \textbf{Scénario Alternatif} & \parbox{0.43\textheight}{
				\begin{enumerate}
					\item \textbf{A1: Les informations de l'acteur ne sont pas valides:}
					\subitem 6. Le site affiche message d'erreur.\newline						
					La séquence résume du point 3.
				\end{enumerate}}\\
			\hline
			\cellcolor[HTML]{99ccff} \textbf{Scénario d'exception} & ---\\
			\hline
			\cellcolor[HTML]{99ccff} \textbf{Post-condition} & La mise-à-jour de profil de l'acteur\\
			\hline
		\end{tabular}}
		\caption{Description Modifier compte}
		\label{table:modifier account}
	\end{table}
	\begin{figure}[H]
		\centering
		\includegraphics[scale=0.45]{ModifierProfil.jpg}
		\caption{Diagramme séquence Modifier Profil}
		\label{fig:seq update account}
	\end{figure}
\end{center}


\subsection{Cas d'utilisation Poster appel d'offre}
\renewcommand{\arraystretch}{2}
\begin{center}
	\begin{table}[H]
		\centering
		\tiny{\begin{tabular}{ | l | m{0.51\textheight}|}
				\hline
				\rowcolor[HTML]{06a8ed}
				\textbf{Nom du cas} & Poster appel d'offre \\
				\hline \hline
				\cellcolor[HTML]{99ccff} \textbf{Type} & Principal\\
				\hline
				\cellcolor[HTML]{99ccff} \textbf{Acteur Principal} & Client\\
				\hline
				\cellcolor[HTML]{99ccff} \textbf{Objective} & Permet au client de poster un offre de travail pour les artisans\\
				\hline
				\cellcolor[HTML]{99ccff} \textbf{Pré-condition} & S'authentifier\\
				\hline
				\cellcolor[HTML]{99ccff} \textbf{Scénario Nominal} & \parbox{0.43\textheight}{
					\begin{enumerate}
						\vspace{0.01\textheight}
						\item L'acteur clique << Poster appel d'offre >>.
						\item Le site affiche un formualire a remplir.
						\item L'acteur remplis les champs.
						\item L'acteur confirme l'appel.
						\item Le système vérifie informations.
						\item Le système crée un appel d'offre.
						\vspace{0.01\textheight}
				\end{enumerate}}\\
				\hline
				\cellcolor[HTML]{99ccff} \textbf{Scénario Alternatif} & \parbox{0.43\textheight}{
					\begin{enumerate}
						\item \textbf{A1: Les informations du client ne sont pas valides:}
							\subitem 6. Le site affiche message d'erreur.\newline						
						La séquence résume du point 3.
						\item \textbf{A2: Le client annule / ne confirme pas l'appel:}
							\subitem 5. Le site retourne a la page d'acceuile
				\end{enumerate}}\\
				\hline
				\cellcolor[HTML]{99ccff} \textbf{Scénario d'exception} & ---\\
				\hline
				\cellcolor[HTML]{99ccff} \textbf{Post-condition} & Un nouveau appel d'offre sera crée\\
				\hline
		\end{tabular}}
		\caption{Description Poster un appel d'offre}
		\label{table:post appel offre}
	\end{table}
	\begin{figure}[H]
		\centering
		\includegraphics[scale=0.45]{PosterAppelOffre.jpg}
		\caption{Diagramme séquence Poster Appel Offre}
		\label{fig:seq post appel offre}
	\end{figure}
\end{center}


\subsection{Cas d'utilisation Soumission appel d'offre}
\renewcommand{\arraystretch}{2}
\begin{center}
	\begin{table}[H]
		\centering
		\tiny{\begin{tabular}{ | l | m{0.51\textheight}|}
				\hline
				\rowcolor[HTML]{06a8ed}
				\textbf{Nom du cas} & Soumission appel d'offre \\
				\hline\hline
				\cellcolor[HTML]{99ccff} \textbf{Type} & Principal \\
				\hline
				\cellcolor[HTML]{99ccff} \textbf{Acteur Principal} & Artisan\\
				\hline
				\cellcolor[HTML]{99ccff} \textbf{Objective} & Permet à artisan de soumettre un appel d'offre à un client\\
				\hline
				\cellcolor[HTML]{99ccff} \textbf{Pré-condition} & S'authentifier\\
				\hline
				\cellcolor[HTML]{99ccff} \textbf{Scénario Nominal} & \parbox{0.43\textheight}{
					\begin{enumerate}
						\vspace{0.01\textheight}
						\item Artisan cliquer sur << Soumettre appel d’offre >>
						\item Le système affiche la liste des offres
						\item L'artisan soumis un ou plusieurs offres
						\item Le système sauvegarde la soumission
						\item Le système notifie le client
						\vspace{0.01\textheight}
				\end{enumerate}}\\
				\hline
				\cellcolor[HTML]{99ccff} \textbf{Scénario Alternatif} & --- \\
				\hline
				\cellcolor[HTML]{99ccff} \textbf{Scénario d'exception} & --- \\
				\hline
				\cellcolor[HTML]{99ccff} \textbf{Post-condition} & Un appel d'offre sera soumis\\
				\hline
		\end{tabular}}
		\caption{Description Soumission d'un appel d'offre}
		\label{table:soumettre offre}
	\end{table}
	\begin{figure}[H]
		\centering
		\includegraphics[scale=0.45]{SoumttreOffre.jpg}
		\caption{Diagramme séquence Soumettre un appel d'offre}
		\label{fig:seq soumettre offre}
	\end{figure}
\end{center}


\subsection{Cas d'utilisation Signaler un utilisateur}
\renewcommand{\arraystretch}{2}
\begin{center}
	\begin{table}[H]
		\centering
		\tiny{\begin{tabular}{ | l | m{0.51\textheight}|}
				\hline
				\rowcolor[HTML]{06a8ed}
				\textbf{Nom du cas} & Signaler un utilisateur \\
				\hline\hline
				\cellcolor[HTML]{99ccff} \textbf{Type} & Principal \\
				\hline
				\cellcolor[HTML]{99ccff} \textbf{Acteur Principal} & Client, Artisan\\
				\hline
				\cellcolor[HTML]{99ccff} \textbf{Objective} & Permet à un utilisateur de signaler un autre utilisateur\\
				\hline
				\cellcolor[HTML]{99ccff} \textbf{Pré-condition} & Consulter Profil\\
				\hline
				\cellcolor[HTML]{99ccff} \textbf{Scénario Nominal} & \parbox{0.43\textheight}{
					\begin{enumerate}
						\vspace{0.01\textheight}
						\item L'acteur clique consulte le profil de l'utilisateur.
						\item Le site affiche le profil de l'utilisateur.
						\item L'acteur clique sur << Signaler >>.
						\item Le système affiche un forumalire a remplir.
						\item L'acteur remplis le formulaire.
						\item Le système vérifie informations.
						\item Le système envoie sauvgarde le rapport et l'envoie aux administrateurs.
						\vspace{0.01\textheight}
				\end{enumerate}}\\
				\hline
				\cellcolor[HTML]{99ccff} \textbf{Scénario Alternatif} & \parbox{0.43\textheight}{
					\begin{enumerate}
						\item \textbf{A1: Les informations du formulaire ne sont pas valides:}
						\subitem 7. Le site affiche message d'erreur.\newline						
						La séquence résume du point 4.
						\item \textbf{A2: Le client annule / ne confirme pas l'appel:}
						\subitem 5. Le site retourne a la page d'acceuile
				\end{enumerate}}\\
				\hline
				\cellcolor[HTML]{99ccff} \textbf{Scénario d'exception} & --- \\
				\hline
				\cellcolor[HTML]{99ccff} \textbf{Post-condition} & Un rapport sera envoyé aux administrateurs\\
				\hline
		\end{tabular}}
		\caption{Description Signaler un utilisateur}
		\label{table:signaler utilisateur}
	\end{table}
	\begin{figure}[H]
		\centering
		\includegraphics[scale=0.44]{SignalerUtilisateur.jpg}
		\caption{Diagramme séquence Signaler un Utilisateur}
		\label{fig:seq signaler utilisateur}
	\end{figure}
\end{center}


\subsection{Cas d'utilisation Bloquer un utilisateur}
\renewcommand{\arraystretch}{2}
\begin{center}
	\begin{table}[H]
		\centering
		\tiny{\begin{tabular}{ | l | m{0.51\textheight}|}
				\hline
				\rowcolor[HTML]{06a8ed}
				\textbf{Nom du cas} & Bloquer un utilisateur \\
				\hline\hline
				\cellcolor[HTML]{99ccff} \textbf{Type} & Principal \\
				\hline
				\cellcolor[HTML]{99ccff} \textbf{Acteur Principal} & Administrateur\\
				\hline
				\cellcolor[HTML]{99ccff} \textbf{Objective} & Permet à un administrateur de bloquer (banner) un utilisateur\\
				\hline
				\cellcolor[HTML]{99ccff} \textbf{Pré-condition} & S'authentifier\\
				\hline
				\cellcolor[HTML]{99ccff} \textbf{Scénario Nominal} & \parbox{0.43\textheight}{
					\begin{enumerate}
						\vspace{0.01\textheight}
						    \item L’acteur visite la fenêtre d’édition
							\item Une liste de tous les utilisateurs sera affichée
							\item L’administrateur clique sur l’utilisateur voulu
							\item L’admin clique sur bouton bloquer
							\item Un message confirmation s’affichera
							\item L’admin confirme son action
							\item Le systeme banne l'utilisateur
						\vspace{0.01\textheight}
				\end{enumerate}}\\
				\hline
				\cellcolor[HTML]{99ccff} \textbf{Scénario Alternatif} & \parbox{0.43\textheight}{
					\begin{enumerate}
						\item \textbf{A1: L'administrateur annule l'action:}
						\subitem 7. Le systeme affiche la page d'acceuille.\newline						
						\item \textbf{A2: L'utilisateur est un autre administrateur:}
						\subitem 7. Le systeme affiche un message d'erreur.
						\subitem 8. Le systeme affiche la page d'acceuille.
				\end{enumerate}}\\
				\hline
				\cellcolor[HTML]{99ccff} \textbf{Scénario d'exception} & --- \\
				\hline
				\cellcolor[HTML]{99ccff} \textbf{Post-condition} & L'utilisateur sera banné\\
				\hline
		\end{tabular}}
		\caption{Description Bloquer un utilisateur}
		\label{table:bloquer utilisateur}
	\end{table}
	\begin{figure}[H]
		\centering
		\includegraphics[scale=0.52]{BloquerUtilisateur.jpg}
		\caption{Diagramme séquence Bloquer un Utilisateur}
		\label{fig:seq bloquer utilisateur}
	\end{figure}
\end{center}


\subsection{Cas d'utilisation Gérer réclamations et signalements}
\renewcommand{\arraystretch}{2}
\begin{center}
	\begin{table}[H]
		\centering
		\tiny{\begin{tabular}{ | l | m{0.51\textheight}|}
				\hline
				\rowcolor[HTML]{06a8ed}
				\textbf{Nom du cas} & Gérer réclamations et signalements \\
				\hline\hline
				\cellcolor[HTML]{99ccff} \textbf{Type} & Principal \\
				\hline
				\cellcolor[HTML]{99ccff} \textbf{Acteur Principal} & Administrateur\\
				\hline
				\cellcolor[HTML]{99ccff} \textbf{Objective} & Permet à un administrateur de Gérer une réclamations et de prendre une action\\
				\hline
				\cellcolor[HTML]{99ccff} \textbf{Pré-condition} & S'authentifier\\
				\hline
				\cellcolor[HTML]{99ccff} \textbf{Scénario Nominal} & \parbox{0.43\textheight}{
					\begin{enumerate}
						\vspace{0.01\textheight}
						\item L’acteur visite la fenêtre des réclamations et signalements
						\item Une liste de tous les réclamations et signalements s’affiche
						\item L’administrateur clique sur une réclamation / signalement
						\item Détailles de la réclamation / signalement s’affiche
						\item L’administrateur, après avoir lu la réclamation, peut soit, en cliquant sur un bouton, l’accepter, la refuser ou contacter la source du réclamation pour plus de détails
						\item En cas d’accepter la réclamation, l’admin sera redirigé vers une nouvelle page (gérer compte utilisateur) pour choisir une action  
						\vspace{0.01\textheight}
				\end{enumerate}}\\
				\hline
				\cellcolor[HTML]{99ccff} \textbf{Scénario Alternatif} & --- \\
				\hline
				\cellcolor[HTML]{99ccff} \textbf{Scénario d'exception} & --- \\
				\hline
				\cellcolor[HTML]{99ccff} \textbf{Post-condition} & La réclamation sera résolu\\
				\hline
		\end{tabular}}
		\caption{Description Gérer réclamations et signalements}
		\label{table:gerer reclamation}
	\end{table}
	\begin{figure}[H]
		\centering
		\includegraphics[scale=0.45]{GererReclamation.jpg}
		\caption{Diagramme séquence Gérer réclamations et signalements}
		\label{fig:seq gere reclamation}
	\end{figure}
\end{center}